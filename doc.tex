\documentclass[%
  reprint,
  aps,
  pra,
  superscriptaddress,
  a4paper,
]{revtex4-2}


\usepackage{enumitem}



\usepackage{authororcid}



% For draft stage
\usepackage{lipsum}

%%%%%%%%%%%%%%%%%%%%%%%%%%%%%%%%%%%%%%%%%%%%%%%%%%%%%%%%%%%%%%%%%%%%%%%%%%%%%%%%%%%%%%%%%%%%%%%%%%%%%%%%%%%%%%%%%%%
%%%%%%%%%%%%%%%%%%%%%%%%%%%%%%%%%%%%%%%%%%%%%%%%%%%%%%%%%%%%%%%%%%%%%%%%%%%%%%%%%%%%%%%%%%%%%%%%%%%%%%%%%%%%%%%%%%%
\usepackage[
  bookmarks=false,
  %bookmarks=true,
  colorlinks,
  linkcolor=blue,
  urlcolor=blue,
  citecolor=blue,
  plainpages=false,
  pdfpagelabels,
  final,
  breaklinks=true
]{hyperref}
\hypersetup{
pdftitle={The Author ORCID package}, 
pdfauthor={Emilio Pisanty}
}
%%%%%%%%%%%%%%%%%%%%%%%%%%%%%%%%%%%%%%%%%%%%%%%%%%%%%%%%%%%%%%%%%%%%%%%%%%%%%%%%%%%%%%%%%%%%%%%%%%%%%%%%%%%%%%%%%%%


\newcommand{\orcidd}[1]{%
  \href{%
    https://orcid.org/#1%
  }{%
   \,\protect\includegraphics[width=8pt]{ORCID-icon.png}\,%
  }%
}


\graphicspath{{./}} % specifies the path for the figures


\begin{document}



\title{The Author ORCID package}


%\author{Emilio Pisanty\,\orcidd{0000-0003-0598-8524}}
\author{Emilio Pisanty}
\orcid{0000-0003-0598-8524}
\email{emilio.pisanty@mbi-berlin.de}
\affiliation{Max Born Institute for Nonlinear Optics and Short Pulse Spectroscopy, Max Born Strasse 2a, D-12489 Berlin, Germany}

\author{Ann Author}
\email{ann.author@institutionality.edu}
\orcid{0000-0000-0000-0000}
\affiliation{The Institute of Institutionality}

\author{Josiah Carberry}
\email{carberry@psychoceramics.edu}
\orcid{0000-0002-1825-0097}
\affiliation{Brown University, Providence, Rhode Island}

\author{Carol Collaborator}
\affiliation{The Institute of Institutionality}
\affiliation{The University of Universality}


\date{\today}



\begin{abstract}
The Author ORCID package is a \LaTeX package for the RevTeX class, which handles the ORCID attribution for authors.
\end{abstract}

\maketitle

%\makeatletter
%\protected@write\@auxout{}{hello3}
%\makeatother

%\newwrite\myfile
%\immediate\openout\myfile=\jobname.foo
%
%\write\myfile{hello}


%\orcid{abc}


%\@orcid


{\color{gray}
\lipsum[1]
}

The command \verb|\orcidlogo{width}| produces the ORCID logo: \orcidlogo{8pt}.


%\doauthor{a}{b}{c}


\setlength{\fboxsep}{0pt}

a\orcidlogo{8pt}b

a\fbox{\orcidlogo{8pt}}b

hah blah


%\addcontentsline{lof}{lof}{thing}

%\listofauthororcid


%\answer{Hard}
%
%\listofanswer

%\write\myoutput{stuff stuff}


%\documentclass[%
  twoside,
  %floatfix,
  reprint,
  amsmath,amssymb,
  aps,
  pra,
  nofootinbib,
  showpacs,
  superscriptaddress,
  a4paper,
  %draft
]{revtex4-1}


\usepackage{graphicx}% 
\usepackage[usenames,dvipsnames]{xcolor}
\usepackage{enumitem}

\usepackage{authororcid}



%\usepackage{pstricks}
%\usepackage[pdf]{pstricks}
\usepackage{tikz}




%%%%%%%%% https://tex.stackexchange.com/a/82318/13423
%
%\def\moveto(#1,#2){%
%  \pgfpathmoveto{\pgfpoint{#1 pt}{#2 pt}}}
%
%\def\curveto(#1,#2)(#3,#4)(#5,#6){%
%  \pgfpathcurveto{\pgfpoint{#1 pt}{#2 pt}}{\pgfpoint{#3 pt}{#4 pt}}{\pgfpoint{#5 pt}{#6 pt}}}
%
%%\let\psset=\tikzset
%\def\pscustom[#1]{\tikzset{#1}}
%\tikzset{
%  xunit/.style={x={(#1,0)}},
%  yunit/.style={y={(0,#1)}},
%  runit/.style={},
%  linecolor/.style={color=#1},
%  linewidth/.style={line width=#1},
%}
%
%\def\newrgbcolor#1#2{\definecolor{#1}{RGB}{#2}}
%\def\pspicture(#1,#2){\tikzpicture}
%\def\endpspicture{%
%\pgfusepath{stroke}
%\endtikzpicture}
%\let\newpath=\relax
%
%











% Use accented characters
\usepackage[utf8]{inputenc}
\usepackage[T1]{fontenc}


% For draft stage
\usepackage{lipsum}

%%%%%%%%%%%%%%%%%%%%%%%%%%%%%%%%%%%%%%%%%%%%%%%%%%%%%%%%%%%%%%%%%%%%%%%%%%%%%%%%%%%%%%%%%%%%%%%%%%%%%%%%%%%%%%%%%%%
%%%%%%%%%%%%%%%%%%%%%%%%%%%%%%%%%%%%%%%%%%%%%%%%%%%%%%%%%%%%%%%%%%%%%%%%%%%%%%%%%%%%%%%%%%%%%%%%%%%%%%%%%%%%%%%%%%%
\usepackage[
  bookmarks=false,
  %bookmarks=true,
  colorlinks,
  linkcolor=blue,
  urlcolor=blue,
  citecolor=blue,
  plainpages=false,
  pdfpagelabels,
  final,
  breaklinks=true
]{hyperref}
\hypersetup{
pdftitle={The Author ORCID package}, 
pdfauthor={Emilio Pisanty}
}
%%%%%%%%%%%%%%%%%%%%%%%%%%%%%%%%%%%%%%%%%%%%%%%%%%%%%%%%%%%%%%%%%%%%%%%%%%%%%%%%%%%%%%%%%%%%%%%%%%%%%%%%%%%%%%%%%%%


\newcommand{\orcid}[1]{%
  \href{%
    https://orcid.org/#1%
  }{%
   \,\protect\includegraphics[width=8pt]{ORCID-icon.png}\,%
  }%
}


\graphicspath{{./}} % specifies the path for the figures
\allowdisplaybreaks



\begin{document}



\title{The Author ORCID package}


\author{Emilio Pisanty\,\orcid{0000-0003-0598-8524}}
\email{emilio.pisanty@mbi-berlin.de}
\affiliation{Max Born Institute for Nonlinear Optics and Short Pulse Spectroscopy, Max Born Strasse 2a, D-12489 Berlin, Germany}


\date{\today}



\begin{abstract}
The Author ORCID package is a \LaTeX package for the RevTeX class, which handles the ORCID attribution for authors.
\end{abstract}

\maketitle

{\color{gray}
\lipsum[1]
}

The command \verb|\orcidlogo{width}| produces the ORCID logo: \orcidlogo{8pt}.

\setlength{\fboxsep}{0pt}

a\fbox{\orcidlogo{8pt}}b

a\orcidlogo{8pt}b

hah blah

























\section*{Author ORCIDs}
\vspace{-1mm}
\begin{itemize}[
  itemsep=-1mm,
  leftmargin=\dimexpr0.3cm+\labelsep\relax,
  label={\smash{\includegraphics[width=8pt]{ORCID-icon.png}}}
  ]
\item Emilio Pisanty:
   \href{https://orcid.org/0000-0003-0598-8524}{0000-0003-0598-8524}
\end{itemize}



\bibliographystyle{arthur} 
\bibliography{references}


\end{document}















space

$\ $

%\input{build/\jobname.ans}


%\write\outputorcid{test write}



%\section*{Author ORCIDs}
%\vspace{-1mm}
%\begin{itemize}[
%  itemsep=-1mm,
%  leftmargin=\dimexpr0.3cm+\labelsep\relax,
%  label={\smash{\includegraphics[width=8pt]{ORCID-icon.png}}}
%  ]
%\item Emilio Pisanty:
%   \href{https://orcid.org/0000-0003-0598-8524}{0000-0003-0598-8524}
%\end{itemize}





%\newwrite\outputstream
%\immediate\openout\outputstream=\jobname.looi
%\immediate\write\outputstream{foo 1}
%\immediate\write\outputstream{foo 2}
%\immediate\write\outputstream{\string\textbf{foo 3}}
%\immediate\closeout\outputstream
 
% This reads myfile.tmp and writes its contents

\input{\jobname.looi}
 
%\newread\inputstream
%\immediate\openin\inputstream=\jobname.looi
%\immediate\read\inputstream to \auxcommand
% 
%This is the first line of the file: \auxcommand
% 
%\immediate\read\inputstream to \auxcommand
% 
%This is the second line of the file: \auxcommand
% 
%\immediate\read\inputstream to \auxcommand
% 
%This is the third line of the file: \auxcommand
% 
%\immediate\closein\inputstream






\bibliographystyle{arthur} 
\bibliography{references}


\end{document}













