\documentclass[%
  twoside,
  %floatfix,
  reprint,
  amsmath,amssymb,
  aps,
  pra,
  nofootinbib,
  showpacs,
  superscriptaddress,
  a4paper,
  %draft
]{revtex4-1}


\usepackage{graphicx}% 
\usepackage[usenames,dvipsnames]{xcolor}
\usepackage{enumitem}

\usepackage{authororcid}



%\usepackage{pstricks}
%\usepackage[pdf]{pstricks}
\usepackage{tikz}




%%%%%%%%% https://tex.stackexchange.com/a/82318/13423

\def\moveto(#1,#2){%
  \pgfpathmoveto{\pgfpoint{#1 pt}{#2 pt}}}

\def\curveto(#1,#2)(#3,#4)(#5,#6){%
  \pgfpathcurveto{\pgfpoint{#1 pt}{#2 pt}}{\pgfpoint{#3 pt}{#4 pt}}{\pgfpoint{#5 pt}{#6 pt}}}

%\let\psset=\tikzset
\def\pscustom[#1]{\tikzset{#1}}
\tikzset{
  xunit/.style={x={(#1,0)}},
  yunit/.style={y={(0,#1)}},
  runit/.style={},
  linecolor/.style={color=#1},
  linewidth/.style={line width=#1},
}

\def\newrgbcolor#1#2{\definecolor{#1}{RGB}{#2}}
\def\pspicture(#1,#2){\tikzpicture}
\def\endpspicture{%
\pgfusepath{stroke}
\endtikzpicture}
\let\newpath=\relax













% Use accented characters
\usepackage[utf8]{inputenc}
\usepackage[T1]{fontenc}


% For draft stage
\usepackage{lipsum}

%%%%%%%%%%%%%%%%%%%%%%%%%%%%%%%%%%%%%%%%%%%%%%%%%%%%%%%%%%%%%%%%%%%%%%%%%%%%%%%%%%%%%%%%%%%%%%%%%%%%%%%%%%%%%%%%%%%
%%%%%%%%%%%%%%%%%%%%%%%%%%%%%%%%%%%%%%%%%%%%%%%%%%%%%%%%%%%%%%%%%%%%%%%%%%%%%%%%%%%%%%%%%%%%%%%%%%%%%%%%%%%%%%%%%%%
\usepackage[
  bookmarks=false,
  %bookmarks=true,
  colorlinks,
  linkcolor=blue,
  urlcolor=blue,
  citecolor=blue,
  plainpages=false,
  pdfpagelabels,
  final,
  breaklinks=true
]{hyperref}
\hypersetup{
pdftitle={The Author ORCID package}, 
pdfauthor={Emilio Pisanty}
}
%%%%%%%%%%%%%%%%%%%%%%%%%%%%%%%%%%%%%%%%%%%%%%%%%%%%%%%%%%%%%%%%%%%%%%%%%%%%%%%%%%%%%%%%%%%%%%%%%%%%%%%%%%%%%%%%%%%


\newcommand{\orcid}[1]{%
  \href{%
    https://orcid.org/#1%
  }{%
   \,\protect\includegraphics[width=8pt]{ORCID-icon.png}\,%
  }%
}


\graphicspath{{./}} % specifies the path for the figures
\allowdisplaybreaks



\begin{document}



\title{The Author ORCID package}


\author{Emilio Pisanty\,\orcid{0000-0003-0598-8524}}
\email{emilio.pisanty@mbi-berlin.de}
\affiliation{Max Born Institute for Nonlinear Optics and Short Pulse Spectroscopy, Max Born Strasse 2a, D-12489 Berlin, Germany}


\date{\today}



\begin{abstract}
The Author ORCID package is a \LaTeX package for the RevTeX class, which handles the ORCID attribution for authors.
\end{abstract}

\maketitle

{\color{gray}
\lipsum[1]
}

\examplecommand


\def\svgwidth{60pt}
\input{ORCIDlogo.tex}

blah hah

%\input{ORCID_iD.tex}


\definecolor{orcidgreen}{rgb}{0.65098041,0.80784315,0.22352941}


\begin{tikzpicture}[scale=0.010]
%\draw[gray, thick] (-1,2) -- (2,-4);
%\draw[gray, thick] (-1,-1) -- (2,2);
%\filldraw[black] (0,0) circle (2pt) node[anchor=west] {Intersection point};

%\draw [red] plot [smooth cycle] coordinates {
%  (256,128) (256,57.3)(198.7,0)(128,0)(57.3,0)(0,57.3)(0,128)(0,198.7)(57.3,256)(128,256)(198.7,256)(256,198.7)(256,128)
%  };


%\draw [red] plot [smooth cycle] coordinates {
%  (256,128) 
%  (256,57.3) .. controls (198.7,0) .. (128,0)
%  };

\fill [orcidgreen]
  (256,128)
    .. controls (256,57.3)  and (198.7,0)   .. (128,0)
    .. controls (57.3,0)    and (0,57.3)    .. (0,128) 
    .. controls (0,198.7)   and (57.3,256)  .. (128,256)
    .. controls (198.7,256) and (256,198.7) .. (256,128)
    ;

\fill [white]
  (86.3,69.8)--(70.9,69.8)
             --(70.9,176.9)
             --(86.3,176.9)
             --(86.3,128.5)
             --(86.3,69.8)
  ;

\fill [white]
  (88.7,199.2)
    .. controls (88.7,193.7) and (84.2,189.1) .. (78.6,189.1)
    .. controls (73,189.1)   and (68.5,193.7) .. (68.5,199.2)
    .. controls (68.5,204.8) and (73,209.3)   .. (78.6,209.3)
    .. controls (84.2,209.3) and (88.7,204.7) .. (88.7,199.2)
  ;

\fill [white]
  (108.9,176.9)
  --(150.5,176.9)
  .. controls (190.1,176.9) and (207.5,148.6) ..(207.5,123.3)
  .. controls (207.5,95.8) and (186,69.7) .. (150.7,69.7)
  --(108.9,69.7)
  --(108.9,176.9)
  (124.3,83.6)
  --(148.8,83.6)
  .. controls (183.7,83.6)  and (191.7,110.1) .. (191.7,123.3)
  .. controls (191.7,144.8) and (178,163)     .. (148,163)
  --(124.3,163)
  --(124.3,83.6)
  ;

%{
%\newrgbcolor{curcolor}{1 1 1}
%\pscustom[linestyle=none,fillstyle=solid,fillcolor=curcolor]
%{
%\newpath
%\moveto(108.9,176.9)
%\lineto(150.5,176.9)
%\curveto(190.1,176.9)(207.5,148.6)(207.5,123.3)
%\curveto(207.5,95.8)(186,69.7)(150.7,69.7)
%\lineto(108.9,69.7)
%\lineto(108.9,176.9)
%\closepath
%\moveto(124.3,83.6)
%\lineto(148.8,83.6)
%\curveto(183.7,83.6)(191.7,110.1)(191.7,123.3)
%\curveto(191.7,144.8)(178,163)(148,163)
%\lineto(124.3,163)
%\lineto(124.3,83.6)
%\closepath
%}
%}

\end{tikzpicture}

hah blah










%\psset{xunit=.25pt,yunit=.25pt,runit=.25pt}
\tikzset{xunit=.25pt,yunit=.25pt,runit=.25pt}
\begin{pspicture}(256,256)
{
\newrgbcolor{curcolor}{0.65098041 0.80784315 0.22352941}
%\pscustom[linestyle=none,fillstyle=solid,fillcolor=curcolor]
{
%\newpath
\moveto(256,128)
\curveto(256,57.3)(198.7,0)(128,0)
\curveto(57.3,0)(0,57.3)(0,128)
\curveto(0,198.7)(57.3,256)(128,256)
\curveto(198.7,256)(256,198.7)(256,128)
%\closepath
}
}
%
%
%
%{
%\newrgbcolor{curcolor}{1 1 1}
%\pscustom[linestyle=none,fillstyle=solid,fillcolor=curcolor]
%{
%\newpath
%\moveto(86.3,69.8)
%\lineto(70.9,69.8)
%\lineto(70.9,176.9)
%\lineto(86.3,176.9)
%\lineto(86.3,128.5)
%\lineto(86.3,69.8)
%\closepath
%}
%}
%
%
%
%{
%\newrgbcolor{curcolor}{1 1 1}
%\pscustom[linestyle=none,fillstyle=solid,fillcolor=curcolor]
%{
%\newpath
%\moveto(108.9,176.9)
%\lineto(150.5,176.9)
%\curveto(190.1,176.9)(207.5,148.6)(207.5,123.3)
%\curveto(207.5,95.8)(186,69.7)(150.7,69.7)
%\lineto(108.9,69.7)
%\lineto(108.9,176.9)
%\closepath
%\moveto(124.3,83.6)
%\lineto(148.8,83.6)
%\curveto(183.7,83.6)(191.7,110.1)(191.7,123.3)
%\curveto(191.7,144.8)(178,163)(148,163)
%\lineto(124.3,163)
%\lineto(124.3,83.6)
%\closepath
%}
%}
%
%
%{
%\newrgbcolor{curcolor}{1 1 1}
%\pscustom[linestyle=none,fillstyle=solid,fillcolor=curcolor]
%{
%\newpath
%\moveto(88.7,199.2)
%\curveto(88.7,193.7)(84.2,189.1)(78.6,189.1)
%\curveto(73,189.1)(68.5,193.7)(68.5,199.2)
%\curveto(68.5,204.8)(73,209.3)(78.6,209.3)
%\curveto(84.2,209.3)(88.7,204.7)(88.7,199.2)
%\closepath
%}
%}
\end{pspicture}

















\section*{Author ORCIDs}
\vspace{-1mm}
\begin{itemize}[
  itemsep=-1mm,
  leftmargin=\dimexpr0.3cm+\labelsep\relax,
  label={\smash{\includegraphics[width=8pt]{ORCID-icon.png}}}
  ]
\item Emilio Pisanty:
   \href{https://orcid.org/0000-0003-0598-8524}{0000-0003-0598-8524}
\end{itemize}



\bibliographystyle{arthur} 
\bibliography{references}


\end{document}













