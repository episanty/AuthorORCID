%% The Author ORCID package
%% Copyright 2020 Emilio Pisanty
%
% This work may be distributed and/or modified under the
% conditions of the LaTeX Project Public License, either version 1.3
% of this license or (at your option) any later version.
% The latest version of this license is in
%   http://www.latex-project.org/lppl.txt
% and version 1.3 or later is part of all distributions of LaTeX
% version 2005/12/01 or later.
%
% This work has the LPPL maintenance status `maintained'.
% 
% The Current Maintainer of this work is Emilio Pisanty. Any updates
% to this work will be posted to https://github.com/episanty/authororcid
%
% This work consists of the files authororcid.sty and authororcid.tex.


\documentclass[%
  reprint,
  aps,
  pra,
  superscriptaddress,
  a4paper,
]{revtex4-2}

\usepackage{authororcid}



%%%%%%%%%%%%%%%%%%%%%%%%%%%%%%%%%%%%%%%%%%%%%%%%%%%%%%%%%%%%%%%%%%%%%%%%%%%%%%%%%%%%%%%%%%%%%%%%%%%%%%%%%%%%%%%%%%%
%%%%%%%%%%%%%%%%%%%%%%%%%%%%%%%%%%%%%%%%%%%%%%%%%%%%%%%%%%%%%%%%%%%%%%%%%%%%%%%%%%%%%%%%%%%%%%%%%%%%%%%%%%%%%%%%%%%
\usepackage[
  bookmarks=false,
  bookmarks=true,
  colorlinks,
  linkcolor=blue,
  urlcolor=blue,
  citecolor=blue,
  plainpages=false,
  pdfpagelabels,
  final,
  breaklinks=true
]{hyperref}
\hypersetup{
pdftitle={The Author ORCID package}, 
pdfauthor={Emilio Pisanty}
}
%%%%%%%%%%%%%%%%%%%%%%%%%%%%%%%%%%%%%%%%%%%%%%%%%%%%%%%%%%%%%%%%%%%%%%%%%%%%%%%%%%%%%%%%%%%%%%%%%%%%%%%%%%%%%%%%%%%



\begin{document}



\title{The Author ORCID package}

\author{Emilio Pisanty}
\orcid{0000-0003-0598-8524}
\email{emilio.pisanty@mbi-berlin.de}
\affiliation{Max Born Institute for Nonlinear Optics and Short Pulse Spectroscopy, Max Born Strasse 2a, D-12489 Berlin, Germany}

\author{Josiah Carberry}
\orcid{0000-0002-1825-0097}
\affiliation{Department of Psychoceramics, Brown University}

\date{\today}

\begin{abstract}
The Author ORCID package is a {\LaTeX} package for the REVTeX class, which handles ORCID iD attribution for authors.
\end{abstract}

\maketitle


This package adds support for ORCID iDs to the REVTeX class. It provides the following features:

\begin{itemize}
\item
An \verb|\orcidicon{width}| command, which produces the ORCID iD icon at the specified width: small, \orcidicon{8pt}, or bigger:

\orcidicon{50pt}

This graphic is produced within \texttt{tikz}, and it is a transcription of the SVG version.
The default width for the icon is 8pt, and it is encoded as the \verb|\orcidiconwidth| length.

\item
An \verb|\orcid{orcid-id}| command, which can be added after each \verb|\author| with the corresponding numerical iD. 
An example usage is as follows:
\begin{verbatim}
\author{Josiah Carberry}
\orcid{0000-0002-1825-0097}
\affiliation{Department of Psychoceramics, 
               Brown University}
\end{verbatim}
This will produce ORCID badges after each author for which an iD has been specified, hyperlinked to that author's ORCID record.


\item
An \verb|\listoforcidids| command, which produces an unnumbered section listing all the author ORCID iDs, suitably hyperlinked, as shown below.

\end{itemize}

This package calls on the \texttt{tikz}, \texttt{xcolor} and \texttt{enumitem} packages as dependences.

To be honest, the code here is quite `hacky' and probably quite fragile.
It is unlikely that this is a sustainable way forward, but the code presented here will hopefully be a useful prototype and temporary fix until full ORCID iD support is included into the REVTeX class itself.

For full clarity, Josiah Carberry is a fictional person and has no involvement with this work---the authorship is wholly mine (E.\ Pisanty), and Carberry (described as an expert in the study of `cracked pots' by \href{https://en.wikipedia.org/wiki/Josiah_S._Carberry}{Wikipedia}) is \href{https://support.orcid.org/hc/en-us/articles/360006897674-Structure-of-the-ORCID-Identifier}{listed by ORCID} as a suitable example author for demonstration purposes.

This package contains minor amounts of code modified or adapted from the REVTeX 4.2 class, and more specifically from its component package ltxfront. Both of these are available from \href{https://journals.aps.org/revtex}{the REVTeX homepage}.

\listoforcidids



\end{document}













